\chapter{Wprowadzenie}

Mechanizm FAMTAR (Flow-Aware Multi-Topology Adaptive Routing) to nowe rozwiązanie zapewniające ruting wielodrogowy i adaptacyjny w sieciach IP. Został on zaproponowany w \cite{FAMTAR_LETTERS} przez opiekunów autora tej pracy. Stanowi on próbę odpowiedzi na wady istniejących mechanizmów rutingu wielodrogowego i adaptacyjnego. Może on współpracować z dowolnym protokołem rutingu wewnątrzdomenowego. Wśród jego zalet można wymienić także w pełni rozproszoną strukturę oraz brak podatności na tzw. oscylację tras.

Mechanizm FAMTAR wydaje się w tej chwili obiecującym rozwiązaniem problemu rutingu wielodrogowego i adaptacyjnego. Wprowadzenie go do powszechnego użytku pozwoliłoby na znaczne zwiększenie efektywności wykorzystania zasobów sieciowych. Jako rozwiązanie nowe, wymaga on jednak szczegółowych badań.

Celem tej pracy było, wobec tego, stworzenie rutera programowego realizującego koncepcję FAMTAR. Został on zaimplementowany w środowisku Click i pracuje pod kontrolą systemu operacyjnego Linux. Oprócz samej implementacji rutera, w ramach pracy przeprowadzone zostały testy sieci zbudowanych z jego użyciem. \\

\noindent Ta praca podzielona jest na sześć rozdziałów. Jej zakres, motywacja i struktura przedstawione zostały w rozdziale pierwszym.

Rozdział \ref{chapter-2} prezentuje tło na jakim powstał mechanizm FAMTAR. Zawarto w nim wprowadzenie do podstaw rutingu IP, po którym przedstawiono problematykę rutingu wielodrogowego i adaptacyjnego. Następnie zawiera on przegląd istniejących rozwiązań w tym zakresie.

Rozdział \ref{chapter-3} przedstawia koncepcję FAMTAR. Opisane są w nim zarówno jej podstawy, jak i rozszerzenia. Część zaprezentowanych tam mechanizmów powstała w wyniku badań autora tej pracy.

Rozdział \ref{chapter-4} opisuje szczegóły implementacji rutera. Zaprezentowano w nim środowisko implementacyjne i uzasadniono jego wybór. Następnie opisano w nim szczegółowo kolejne komponenty rutera.

Rozdział \ref{chapter-5} prezentuje wyniki testów zaimplementowanego rutera, przeprowadzonych w laboratorium sieciowym.

Ostatni rozdział zawiera podsumowanie pracy. Przedstawione są osiągnięte cele i rezultaty.
